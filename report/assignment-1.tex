%%% Local Variables: 
%%% mode: latex
%%% TeX-master: t
%%% End:

\documentclass[a4paper, 10pt]{article}

\usepackage[utf8]{inputenc}
\usepackage{hyperref}

\usepackage[backend=bibtex]{biblatex}
\addbibresource{references.bib}

\title{Software Reengineering\\
       Assignment 1}
\author{Joey Ezechiels (1338994) \and Volker Lanting (1513273)}

% % Implement x.yy numbering scheme for subsections
% % where yy is *always* a 2-digit number
% \makeatletter
% \renewcommand\thesubsection{\thesection.\two@digits{\arabic{subsection}}}
% \makeatother

\begin{document}
\maketitle % Doesn't count towards the total number of pages
\pagenumbering{roman}

\newpage
\tableofcontents % Doesn't count towards the total number of pages

% \newpage
% \section{Introduction}
% \label{sec:introduction}

\newpage
\pagenumbering{arabic}
\section{Initial Questions}
\label{sec:initial_question}
In reverse-engineering the JMonkeyEngine, there are some initial
questions that need answering in order to gain a clearer picture
of what the architecture and code look like.

% For each of these pieces of code, reasons why they violate the
% principles *must* be provided. We must use figures and/or metrics to
% underline our reasoning.

% Simple shortcomings in the source code don't count, such as code
% clones of 1-2 lines, naming, or missing comments.

\subsection{Main features of the program}
\label{sec:main_features}
% What are the main features of the program?



\subsection{Important source code entities}
\label{sec:important_src_entities}
% Which are the important source code entities?


\subsection{Quality, Design and Implementation: First impressions}
\label{sec:first_impressions}
% What is your first impression of the quality of the design and
% implementation (also think of documentation, tests, etc.)?


\subsection{Reengineering feasibility}
\label{sec:reengineering_feasibility}
% Do you think a reengineering is feasible?


\subsection{Exceptional entities}
\label{sec:exceptional_entities}
% What are the exceptiona packages, classes and methods?


\subsection{Inheritance structure}
\label{sec:inheritance_structure}
% What does the inheritance structure look like?


\subsection{Scene composition: Basic elements}
\label{sec:scene_composition}
% What are the basic elements to compose a scene?


\subsection{Scene rendering}
\label{sec:scene_rendering}
% How is a scene rendered?


\subsection{Collision detection}
\label{sec:collision_detection}
% How are collisions detected?


\newpage
\section{Problem detection}
\label{sec:problem_detection}

By utilising a number of tools available to us (see section
\ref{sec:used_tools} for details) and knowledge gained from this
course we have identified a number of shortcomings in both the design
and implementation of JMonkeyEngine. In searching for these
shortcomings we have kept in mind the  S.O.L.I.D. design principles,
though we have also considered the ADP and DRY principles.

\subsection{Single Responsibility violation}
\label{sec:srp_violation}
% 2 examples of an Single Responsibility Principle violation


\subsection{Liskov Substitution Principle violation}
\label{sec:lsp_violation} % TODO if we choose OCP instead this should
                          % be changed to sec:ocp_violation
% one example of an Open/Closed Principle violation or 
% a Liskov substitution Principle violation



\subsection{Dependency Inversion Principle violation}
\label{sec:dip_violation}
% One example of a Dependency Inversion Principle violation


\subsection{Acyclic Dependency Principle violation}
\label{sec:adp_violation}
% One example of an Acyclic Dependency Principle violation


\subsection{Don't Repeat Yourself violation}
\label{sec:dry_violation}
% One example of a Don't Repeat Yourself violation


\subsection{Other class and package design violations}
\label{sec:other_violations}
% This one is optional; If we have the time I recommend we do it - Joey

\newpage
\section{Used tools}
\label{sec:used_tools}


\subsection{static analysis}
\label{sec:static_analysis}


\subsection{Dynamic analysis}
\label{sec:dynamic_analysis}



\end{document}
