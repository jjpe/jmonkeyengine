%%% Local Variables: 
%%% mode: latex
%%% TeX-master: t
%%% End:

\documentclass[a4paper, 10pt]{article}

\usepackage[utf8]{inputenc}
\usepackage{hyperref}
\usepackage{enumerate}

\usepackage[backend=bibtex]{biblatex}
\addbibresource{references.bib}

\title{Software Reengineering\\
       Assignment 1}
\author{Joey Ezechiels (1338994) \and Volker Lanting (1513273)}

% % Implement x.yy numbering scheme for subsections
% % where yy is *always* a 2-digit number
% \makeatletter
% \renewcommand\thesubsection{\thesection.\two@digits{\arabic{subsection}}}
% \makeatother

\begin{document}
\maketitle % Doesn't count towards the total number of pages
\pagenumbering{roman}

\newpage
\tableofcontents % Doesn't count towards the total number of pages

% \newpage
% \section{Introduction}
% \label{sec:introduction}

\newpage
\pagenumbering{arabic}
\section{Initial Questions}
\label{sec:initial_question}
In reverse-engineering the JMonkeyEngine, there are some initial
questions that need answering in order to gain a clearer picture
of what the architecture and code look like.

% For each of these pieces of code, reasons why they violate the
% principles *must* be provided. We must use figures and/or metrics to
% underline our reasoning.

% Simple shortcomings in the source code don't count, such as code
% clones of 1-2 lines, naming, or missing comments.

\subsection{Main features of the program}
\label{sec:main_features}
% What are the main features of the program?

jMonkeyEngine can be used to create games. To this end it offers a
number of features, among which we consider the following ones the
most important:
\begin{itemize}
	\item Creating Applications 
	(\verb|com.jme3.app|, \verb|com.jme3.system|)
	\item Managing Assets of these applications (\verb|com.jme3.asset|)
	The main assets are scenes, 
	located in the \verb|com.jme3.scene| package.
	% \begin{itemize}
	% 	\item Animations (\verb|com.jme3.animation|)
	% 	\item Meshes (\verb|com.jme3.scene|)
	% 	\item Textures (\verb|com.jme3.texture|)
	% 	\item 2D Pictures (\verb|com.jme3.ui|)
	% 	\item Scenes
	% 	(referred to as Spatial in the code \verb|com.jme3.scene|)
	% 	\item Materials (\verb|com.jme3.material|)
	% 	\item Shaders (\verb|com.jme3.shader|)
	% \end{itemize}
	
	\item Real time Rendering (\verb|com.jme3.renderer|)
	% \item Creating Particle effects (\verb|com.jme3.effect|)
	% \item Creating post processing effects (\verb|com.jme3.post|)
	\item Collision detection (\verb|com.jme3.collision|, \verb|com.jme3.bounding|)
	% \item Cinematic events and motion paths (\verb|com.jme3.cinematic|)
	% \item Exporting assets (\verb|com.jme3.export|)
	% \item Support for fonts (\verb|com.jme3.font|, \verb|com.jme3.renderer|)
	% \item Dealing with user input (\verb|com.jme3.input|)
	% \item Lighting of a scene (\verb|com.jme3.light|, \verb|com.jme3.shadow|)
	\item Shaders (\verb|com.jme3.shader|)
\end{itemize}


\subsection{Important source code entities}
\label{sec:important_src_entities}
% Which are the important source code entities?
For managing assets, the AssetManager and ImplHandler are important.

For creating applications, the SimpleApplication is important.
For managing and configuring them, the
AppSettings, JmeSystem and JmeSystemDelegate, 
AbstractAppState and AppStateManager 
seem to be important.

For real time rendering, the  RenderManager,
Camera, ViewPort and
RenderQueue are important.
The actual Renderer is located in the lwjgl subsystem 
and not part of the core subsystem. 
Therefore we will skip them for this project.
 
For shaders, the Shader, Uniform and UniformBindingManager seem important.
% Volker

\subsection{Quality, Design and Implementation:\\First impressions}
\label{sec:first_impressions}
% What is your first impression of the quality of the design and
% implementation (also think of documentation, tests, etc.)?

% Joey

% Notes:
% - package structure is good
% - The code itself uses a lot of interfaces but also
%   shows signs of organic growth.
% - There is a tradeof between speed and design,
%   epecially in the com.jme3.math package

\subsection{Reengineering feasibility}
\label{sec:reengineering_feasibility}
% Do you think a reengineering is feasible?

% Volker

\subsection{Exceptional entities}
\label{sec:exceptional_entities}
% What are the exceptional packages, classes and methods?

% Volker (sonar):
% - packages
% - Search for methods ans classes that smell

% Joey (inCode):
% - god classes and general class & interface stuff
% - brain methods & feature envy

\subsection{Inheritance structure}
\label{sec:inheritance_structure}
% What does the inheritance structure look like?

% Volker (& Joey)

\subsection{Scene composition: Basic elements}
\label{sec:scene_composition}
% What are the basic elements to compose a scene?

% Volker

\subsection{Scene rendering}
\label{sec:scene_rendering}
% How is a scene rendered?

% Joey

\subsection{Collision detection}
\label{sec:collision_detection}
% How are collisions detected?

% in the com.jme3.bounding package for collision detection
% Bounding boxes are being made, check the code and the site
% for details

\newpage
\section{Problem detection}
\label{sec:problem_detection}

By utilising a number of tools available to us (see section
\ref{sec:used_tools} for details) and knowledge gained from this
course we have identified a number of shortcomings in both the design
and implementation of JMonkeyEngine. In searching for these
shortcomings we have kept in mind the  S.O.L.I.D. design principles,
though we have also considered the ADP and DRY principles.

\subsection{Single Responsibility violation}
\label{sec:srp_violation}
% 2 examples of an Single Responsibility Principle violation


\subsection{Liskov Substitution Principle violation}
\label{sec:lsp_violation} % TODO if we choose OCP instead this should
                          % be changed to sec:ocp_violation
% one example of an Open/Closed Principle violation or 
% a Liskov substitution Principle violation



\subsection{Dependency Inversion Principle violation}
\label{sec:dip_violation}
% One example of a Dependency Inversion Principle violation


\subsection{Acyclic Dependency Principle violation}
\label{sec:adp_violation}
% One example of an Acyclic Dependency Principle violation


\subsection{Don't Repeat Yourself violation}
\label{sec:dry_violation}
% One example of a Don't Repeat Yourself violation


\subsection{Other class and package design violations}
\label{sec:other_violations}
% This one is optional; If we have the time I recommend we do it - Joey

\newpage
\section{Used tools}
\label{sec:used_tools}

Below is a list of the analysis and refactoring tools we used.
% separated into \emph{static analysis} and \emph{dynamic analysis}.

% \subsection{Static analysis}
% \label{sec:static_analysis}

\begin{enumerate}[i)]
\item \textbf{inCode} is an Eclipse plugin that generates \emph{class
    blueprints} from Java source code.
\item \textbf{Sonar}
\end{enumerate}

% \subsection{Dynamic analysis}
% \label{sec:dynamic_analysis}

% \begin{enumerate}[i)]
% \item \textbf{}
% \end{enumerate}

\end{document}
